\documentclass[12pt]{niuthesis}
\usepackage[utf8]{inputenc}
\usepackage{graphicx}
\usepackage{amsmath}
\usepackage{physics}
\usepackage{caption}
\usepackage{subcaption}
\usepackage{multirow}
\usepackage{xcolor}
\usepackage{listings}
\usepackage{braket}
\usepackage{hyperref}
\usepackage{pdflscape}
\usepackage{soul}
\usepackage{setspace}
\usepackage{cleveref}
\usepackage[T1]{fontenc}
\usepackage{array}
\usepackage{float}

\usepackage{listings}
\usepackage{xcolor}

% OMIT FOR FINAL PRODUCT
\usepackage{lineno}
%\linenumbers

\definecolor{codegreen}{rgb}{0,0.6,0}
\definecolor{codegray}{rgb}{0.5,0.5,0.5}
\definecolor{codepurple}{rgb}{0.58,0,0.82}
\definecolor{backcolour}{rgb}{0.95,0.95,0.92}

\lstdefinestyle{mystyle}{
    backgroundcolor=\color{backcolour},   
    commentstyle=\color{codegreen},
    keywordstyle=\color{magenta},
    numberstyle=\tiny\color{codegray},
    stringstyle=\color{codepurple},
    basicstyle=\ttfamily\footnotesize,
    breakatwhitespace=false,         
    breaklines=true,                 
    captionpos=b,
    frame=single,                    
    keepspaces=true,                 
    numbers=left,                    
    numbersep=5pt,                  
    showspaces=false,                
    showstringspaces=false,
    showtabs=false,                  
    tabsize=2
}

\lstset{style=mystyle}

\usepackage[backend=biber,backref,sorting=none]{biblatex}
\addbibresource{main.bib}

\title{RNTuple for ATLAS Analysis Workflows}
\author{Fatima Rodriguez}
\major{Physics}
\degree{Thesis}{M.S.}{Master of Science}
\degreedate{December}{2025}
\department{Department of Physics}
\director{Hector de la Torre Perez}

\begin{document}

\begin{abstract} 
    RNTuple is the new data storage format set to replace TTree at the start of the High-Luminosity LHC. An investigation was conducted to evaluate how analysis workflows for ATLAS researchers will change with RNTuple, using reading speed, writing speed, disk space, and memory consumption as metrics. In this study, all metrics were measured using converted RNTuple inputs from ATLAS Open Data, compared to their TTree equivalents. Additionally, RNTuples produced with the \texttt{LZ4} compression algorithm were generated and compared with those produced using \texttt{ZSTD}. Finally, two new versions of the Analysis Grand Challenge (AGC) using ATLAS Open Data were completed for TTree and RNTuple inputs with RDataFrame in Python. This constitutes the first implementation of an end-to-end analysis completed using RNTuple.  
\end{abstract}

\begin{dedication}
  Para mis padres, con toda mi gratitud por su amor y apoyo.
\end{dedication}

\begin{acknowledgments}
  I would first like to thank my advisors: Hector, Peter, Walter, and Serhan. I feel incredibly fortunate to have had outstanding advisors over the years. I am deeply grateful for your guidance, support, and patience throughout this project. I am also very grateful for my family: Haydee, Jose Alfredo, Saul and Vivianita Rodriguez. I am very proud to be your daughter and older sister. Thank you all for encouraging me to be brave and to pursue this passion. Los quiero mucho! Many thanks to my fanbase: Mariela, Audrey, Jorge, Susan, Syriah, the Garcia Family, Emily, et al. I am so lucky to be surrounded by your beautiful friendships! Thank you all for hyping me up throughout my career. Thank you to my big sister and mentora, Dr. Arcelia Hermosillo! Meeting you was significant not only for my career but also for the friendship that I will always cherish. Thank you so much to Dr. Brendan Kiburg, Dr. Saskia Charity, and Fermilab friends for bringing me to Illinois and solidifying my passion in high energy physics! Big shout out to Net Force, the PGSA, and colleagues. It was an honor to serve as your volleyball coach! Thank you for going along with my social event ideas. As one can see, it took many communities for this accomplishment. Additional shout outs to those that propelled me here and continue to support others like me: Cal NERDs, Hispanic Engineers and Scientists, and the Latin American Graduate Student Association. Finally, I would like to acknowledge $(CP)^2$ for funding the work done in this thesis. 
\end{acknowledgments}
\MakeThesisPrologue 

\chapter{Introduction}
\label{firstchapter}
\chapter{Introduction}		% chapter 1
\label{introchap}		% for reference (\ref{introchap})
% what is particle physics and why is it important to the world? 
Our current understanding of the building blocks of our universe is summarized with one model, called the Standard Model (SM). From the way we power our cities, to the particles that hold them together, the SM explains how the basic building blocks of matter interact, governed by the four fundamental forces: gravity, electromagnetism, the strong force and the weak force. Yet, questions remain about the SM, such as why are there only three generations of fundamental particles? What is the nature of dark matter and dark energy, and how does it fit within the SM? What about the origin of the matter-antimatter asymmetry? Is there a unification theory for the fundamental forces? Is the SM complete or do other exotic particles exists? Over the years, experimental particle physicists and engineers have built technology to test the SM, either by performing precision measurements of particles and their behaviors, or by colliding particles and measuring their outputs. As a result, we have increased our confidence in the SM theory, but continue to search for answers for these remaining questions through experimental discovery.  
% add SM figure
\begin{figure}
\begin{subfigure}{0.5\textwidth}
\includegraphics[width=0.9\linewidth, height=6cm]{ch1_images/SMSummary.png}
\end{subfigure}
\begin{subfigure}{0.5\textwidth}
\includegraphics[width=0.9\linewidth,height=6cm]{ch1_images/SMSummary_b.png}
\end{subfigure}
\caption[Summary of SM cross-section Measurments]{
	Summary of several Standard Model cross-section measurements (a) with associated references (b) \cite{ATLASSummaryPlots}. Cross-sections can be thought of as probabilities that a process occurs. This means that the processes with smaller cross-sections are considered rare-processes because it has a lower probability of being observed. Increasing the probability of these rare-processes would require an increase of luminosity or collisions. The measurements are corrected for branching fractions, compared to the corresponding theoretical expectations. 
	}
\label{fig:SM_cross-sections}
\end{figure}

A Toroidal LHC Apparatus (ATLAS) is a particle physics experiment designed to detect the high-energy particle collisions from the Large Hadron Collider (LHC). At the LHC, collisions take place at a rate of more than a billion interactions per second, which is a combined data volume of about 60 million megabytes per second. However, in order to study rare processes, as shown in Figure \ref{fig:SM_cross-sections}, the LHC will have a major upgrade to increase the number of collisions by a factor of 5 to 7.5. This upgrade, called the High-Luminosity LHC, will require a new data storage format that can handle this increase in data.
% I need to edit the wording in this paragraph

%%%Introduce what RNTuple is and why it will be replacing TTree.
RNTuple is the new ROOT data storage format that will be in use at the start of the HL-LHC. Due to its design, which takes advantage of modern C++ techniques, it is set to improve read speedability and memory usage compared to its predecessor, TTree, and other data storage formats such as HDF5 and Parquet. At the start of this work, performance studies on RNTuple were conducted at the production level, and RNTuple was still at an experimental stage. 

%%%Paragraph on paper outline.
This thesis investigates the performance of RNTuple for ATLAS analysis workflows. This chapter will provide a more detail introduction of the SM, followed by an introduction to the ATLAS experiment and its detector technology in Chapter 2. In Chapter 3, the ATLAS software and computing system, and data contents are introduced. In Chapter 4, an introduction to RNTuple and TTree is provided along with examples of how RNTuple is applied in comparison to TTree. Performance studies conducted for RNTuple and how they compare with TTree will be presented in Chapter 5. In Chapter 6, the Analysis Grand Challenge (AGC) is introduced along with its RNTuple implementation. A final discussion and conclusions are given in Chapter 7.

\subsection{Standard model of particle physics}
% one paragraph providing an overview of particle characteristics for matter vs. forces., units, bosons vs fermions. 
The SM is a quantum field theory that explains and catagorizes all observed fundamental particles by their properties and interactions. Quantum field theory (QFT) is the main theoretical tool for describing particle interactions by combining special relativity and quantum mechanics. Due to this combination, QFT is a probabilistic theory where each particle has an associated field that permeates all of space; therefore, forces are simply the interactions between these different fields. For example, the electromagnetic force is just the interaction between the electromagnetic field and charged matter fields, which fall under quantum electrodynamics (QED). In sum, the SM encompasses all known elementary particle interactions, except for gravity, through a collection of quantum field theories, each dictated by gauge symmetries: QED ($U(1)$), the Glashow-Weinberg-Salam theory of electroweak processes ($SU(3)$), and quantum chromodynamics ($SU(2) x U(1)$).

\subsubsection{Symmetries and Particle Content}
In physics, symmetries are fundamental because they lead to conservation laws through Noether's Theorem. Symmetries can manifest in two notions: invariance and covariance. Properties of a system are described as invariant if they do not change under a symmetry transformation. For example, rotating a sphere and without altering gravitational force would indicate a conservation of angular momentum. In contrast, covariance is used to describe a system that changes in accordance to changes induced by symmetry transformations. 

The SM is a gauge theory based on the symmetry group $SU(3)_C x SU(2)_L x U(1)_Y$. Gauge theory is a QFT that requires invariance under continous transformations, and a symmetry group is a set objects that obey the four properties listed in Table \ref{group_axioms}. $SU(3)_c$ is called the color symmetry group describing the strong nuclear force, which is the intraction between quarks and gluons. $SU(2)_L x U(1)_Y$ describes the electromagnetic and weak nuclear forces, which is the interactions between leptons, photons and W\/Z bosons. 
% axioms of a group from Robert Mann book
\begin{table}[htb]
\caption[Properties of a Group.]{\label{group_axioms}
Properties of a Group \cite{Mann:2010nvj}.
}
\begin{center}
\begin{tabular}{c c c}
\hline
CLOSURE & If $g_1, g_2 \in G \rightarrow g_1 \diamond g_2 \in G$ & (combinations remain in the set) \\
IDENTITY & There exists $I \in G \rightarrow I \diamond g_i = g_i$ for every $g_i \in G$ & (one element does nothing) \\
INVERSE & Every $g_i \in G$ has a $g_i^{-1} \in G$ such that $g_1 \diamond g_i^{-1} = I$ & (combinations can be undone) \\
ASSOCIATIVITY & If $g_1, g_2, g_3 \in G \rightarrow (g_1 \diamond g_2) \diamond g_3 = g_1 \diamond (g_2 \diamond g_3)$ & (combinational groupings can be interchanged) \\ 
\hline
\end{tabular}
\end{center}
\end{table}

Furthermore, the four groups of particles shown in Figure \ref{fig:SM_wikid}: quarks, leptons, gauge bosons, and scalar bosons, can be further categorized as \emph{bosons} or \emph{fermions} because of a fundamental property called spin. Similar to the Earth, particles carry orbital angular momentum and spin angular momentum; however, for particles, spin is an intrinsic property. All bosons carry an integer spin; meanwhile, fermions carry half-integer spin. As a result from QFT, each fermion has an antiparticle with the same mass and lifetime as the particle itself, but oppositely charged. The three charged leptons ($e$, $\mu$, $\tau$) are massive, while their corresponding nuetrinos ($\nu_e$,$\nu_{\mu}$,$\nu_{\tau}$), are massless with nuetral charge. Due to QCD, there are 8 types of gluons. The Higgs boson has its own section as a scalar boson because unlike the vector bosons with spin 1, the Higgs boson has spin 0. In sum, there are a total of 12 leptons including their antiparticles, 36 quarks including all the flavors and their antiparticles, 12 vector bosons, and 1 scalar boson, which makes a total of 61 fundamental particles.
% Standard Model Figure
\begin{figure}
\centerline{\includegraphics[height=95mm]{ch1_images/Standard_Model_of_Elementary_Particles.png}}
\caption[The SM.]{Particle content of the Standard Model \cite{SM-wikidpedia}.}
\label{fig:SM_wikid}
\end{figure}

\subsection{Standard Model Limitations}



\subsection{Phenomenology of Large Hadron Colliders}


%%% Local Variables: 
%%% TeX-master: "mythesis"
%%% End: 


\chapter{The ATLAS Experiment}
\label{secondchapter}
ATLAS is a general-purpose experiment, optimized to search for the Higgs boson, top quark decays, and supersymmetry. In July 1997, the ATLAS Experiment was approved and by November 2008, ATLAS was the largest detector ever constructed at 44 meters long and 25 meters in diameter \cite{CERN_Timeline_143}. By November 2009, ATLAS recorded its first proton-proton collision \cite{ATLAS_FirstCollisions2009} and by December 2010, ATLAS observed its first top quark pairs, which are the heaviest known elementary particle with a strong coupling to the Higgs boson \cite{2011}. By July 2012, both ATLAS and the Compact Muon Spectrometer (CMS) experiment successfully observed the Higgs boson \cite{Aad_2012, Chatrchyan_2012}. ATLAS is projected to continue operation until 2041 to continue searching for standing questions from the SM. 

This chapter will provide a brief description of the LHC and the Run 2 ATLAS detector, relevant to the data used in the remainder of this study. 

\section{The Large Hadron Collider}
The LHC is a two-ring-superconducting-hadron accelerator and collider built outside of Geneva, Switzerland at the Conseil Europeen pour la Recherche Nucleaire (CERN) \cite{Lyndon_Evans_2008}. It was approved for construction in 1996 to search for beyond the SM physics the  at energies larger than 10 TeV. It's approval was heavily influenced by the cost-saving idea of reusing the existing 26.5 km tunnels from the Large Electron-Positron (LEP) collider. The LHC has four main collision points that house the ATLAS, CMS, Large Hadron Collider beauty (LHCb) \cite{The_LHCb_Collaboration_2008}, and A Large Ion Collider Experiment (ALICE) \cite{ALICE_Experiment_2008}. ATLAS and CMS are the two high-energy experiments located at diametrically opposite straight sections. LHCb is a low luminosity experiment dedicated to investigate the difference between matter and anti-matter by detecting b quarks. ALICE is an ion experiment dedicated to studying quark-gluon plasma forms. 

The LHC is initially supplied with protons from the injector complex, which is a sequence of accelerators shown in Figure \ref{fig:CERN-complex}. The three main components within each of these accelerators are magnets, vacuum chambers, and radiofrequency (RF) cavities. Superconducting magnets are responsible for guiding the beams, and vacuum chambers ensure that particles do not interact with external residual gas molecules. RF cavities are metallic chambers located inside the beam vacuum. They are designed to resonate at specific frequencies to provide small energy boosts when particles pass through. 
\begin{figure}[ht]
\centerline{\includegraphics[height=95mm]{ch2_images/CCC-v2022.png}}
\caption[The CERN Accelerator Complex]{The CERN accelerator complex \cite{CERN_AcceleratorComplex}.}
\label{fig:CERN-complex}
\end{figure}

During Run 2, the LHC collided protons at a center of mass energy of $\sqrt{s} = 13$ TeV with a combined integrated luminosity of $L = 140$ $fb^{-1}$ \cite{ATLAS:2022hro}. Beams were delivered in bunches with bunch separation of 25 ns, corresponding to a bunch crossing frequency of 40 MHz. 

\section{The ATLAS Apparatus}
The ATLAS detector, shown in Figure \ref{fig:ATLAS-detector}, consists of a collection of subsystems confined in a 46 m long, 25 m in diameter cylinder, 100 m below ground. The first subsystem is the Inner Detector (ID) \cite{2020}, which is responsible for tracking charged-particles. A calorimeter system follows and measures the energy loss of the particles passing through the detector \cite{Starz:2628123}. The final subsystem is the Muon Spectrometer (MS) \cite{Aad_2020}, which measures the deflection of muons within a magnetic field using a trigger and high precision tracking chambers. Additionally, a first-level and high-level trigger system is implemented to select interesting events and record them to disk \cite{Panduro_Vazquez_2017}. 
\begin{figure}[ht]
\centerline{\includegraphics[height=95mm]{ch2_images/0803012_01.jpg}}
\caption[The ATLAS Detector]{Computer generated image of the whole ATLAS detector \cite{ATLAS_Schematics}.}
\label{fig:ATLAS-detector}
\end{figure}

ATLAS uses a cylindrical coordinate system $(r, \eta, \phi)$ for detector design, reconstruction and data analysis. The polar coordinates, $(r, \phi)$, point in the plane towards the center of the LHC ring and upwards. The pseudorapidity, $\eta$, is defined in Equation \ref{eq:pseudorapidity}, where $\theta$ is the polar angle and equal to the true rapidty defined in Equation \ref{eq:truerapidity}.
\begin{equation}
	\eta = -\ln(\tan{\frac{\theta}{2}})
\label{eq:pseudorapidity}
\end{equation}
\begin{equation}
	y = \frac{1}{2}\ln{(\frac{E + p_z}{E - p_z})}
\label{eq:truerapidity}
\end{equation}
The ID tracks particles in the range $|\eta| < 2.5$, the calorimeter system covers $|\eta| < 4.9$, and the MS detects muon in the $|\eta| < 2.7$ range. 

\subsection{The Inner Detector}
The main components of the ID are the Pixel Detector, Semiconductor Tracker (SCT), and the Transition Radiation Tracker (TRT). This layout is provided in Figure \ref{fig:ID}. The Pixel Detector is first to pick up the energy deposits of the collisions at a precision of 10 $\mu m$. Their signals determine the origin and momentum of the particles. The SCT surrounds the Pixel Detector, which measures particle tracks with a precision of up to 25 $\mu m$. The TRT is the final layer that provides particle type information, in combination with the other information gained in the ID.
\begin{figure}[ht]
\centerline{\includegraphics[height=95mm]{ch2_images/0803014_01.jpg}}
\caption[ATLAS Inner Detector Schematics]{Computer generated image of the ATLAS inner detector \cite{ATLAS_Schematics}.}
\label{fig:ID}
\end{figure}

\subsection{Calorimeter Systems}
Calorimeters are detectors that measure the energies and positions of charged and neutral electromagnetically or strongly interacting particles. They consists of highly-dense materials that force particles to deposit their energy. That energy is then converted into a measurable signal using layers of "active" media. The calorimeter systems consists of two types of calorimeters as shown in Figure \ref{fig:calorimeters}: electromagnetic and hadronic. Electromagnetic calorimeters are used to measure charged particles like electrons, positrons, and photons. Hadronic calorimeters are designed to detect hadrons, such as quarks, protons, and neutrons. 
\begin{figure}[ht]
\centerline{\includegraphics[height=95mm]{ch2_images/0803015_01.jpg}}
\caption[ATLAS Calorimeter System]{Computer generated image of the ATLAS calorimeter \cite{ATLAS_Schematics}.}
\label{fig:calorimeters}
\end{figure}

\subsection{Muon Spectrometer}
The muon spectrometer, shown in Figure \ref{fig:spectrometer}, is the outer part of the ATLAS detector, designed to measuring the momentum of muons. Muons are minimally ionizing particles, meaning they can travel to the edge and beyond the ATLAS detector. The magnetic field that bends their directories is generated by superconducting air-core toroidal magnets, located at the two end caps and one in the center barrel. Three stations of precision chambers, consisting of layers of Monitored Drift Tubes (MDTs) detect the deflection of the muon trajectories in the magnetic field. The MDTs allow muons to knock out electrons from gas when passing through, to produce a signal. Two chambers sit surrounding the central region and ends of the experiment: the Resistive Plate Chambers (RPCs) and Thin Gap Chambers (TGCs). They both detect muons when they ionize the gas mixtures to generate signal. 
\begin{figure}[ht]
\centerline{\includegraphics[height=95mm]{ch2_images/0803017_01.jpg}}
\caption[ATLAS Muon Spectrometer]{Computer generated image of the ATLAS Muons subsystems \cite{ATLAS_Schematics}.}
\label{fig:spectrometer}
\end{figure}

\subsection{Magnet System}
The two main magnet systems are the Central Solenoid Magnet and the Toroid Magnets. Generally, superconducting magnets are required to bend the directories of charged particles, allowing for the ATLAS detector to to measure their momentum and charge. The Central Solenoid Magnet  provides a 2 Tesla magnetic field surrounding the inner detector. The Toroid Magnets are located at the ends of the experiment, and a massive toroid magnet surrounds the center of the experiment. As mentioned in the previous section, the magnets at the ends of the experiment are to bend muons for the MS.

\subsection{ATLAS Trigger System}
The ATLAS Trigger system is a collection of electronics that make rapid decisions of saving certain events into disk. There are two trigger subsystems that help selectively read out and store data from interesting physics events. The first level of the trigger system, called the L1 trigger, uses reduced-granularity information from the calorimeters and muon system to search for signatures of these events. The maximum L1 accept rate is 100 kHz, meaning all processing for an event must be completed within that time window. The second level of the trigger system, called the High Level Trigger, is a software-based system that performs a more thorough reconstruction of the events passed in L1 to then finally pass to a data storage system for offline analysis. 

\section{HL-LHC}
The HL-LHC was proposed in 2010 to extend the discovery potential of the LHC by increasing its instantaneous luminosity (rate of collisions) by a factor of five beyond the original design value and the integrated luminosity (total number of collisions) by a factor ten. Increasing the total number of collisions will increase the probability for ATLAS and CMS to observe rare processes at higher precision \cite{ZurbanoFernandez:2020cco}. The HL-LHC configuration relies on innovations in accelerator technology such as cutting edge 11 to 12 Tesla superconducting magnets, novel magnet designs, compact superconducting RF cavities for beam rotation with phase control, new technologies and materials for beam collimation, and high-current superconducting links with almost zero energy dissipation. 

The ATLAS experiment will also require an upgrade following the HL-LHC. New sub-detectors will be installed such as the Inner Tracker \cite{ATLAS-TDR-25, ATLAS-TDR-30}, the High Granularity Timing Detector \cite{ATLAS-TDR-31}, and additional Muon chambers \cite{ATLAS-TDR-26}. There will also be different electronics upgrades such as the Liquid Argon Calorimeter \cite{ATLAS-TDR-27}, the Tile Calorimeter \cite{ATLAS-TDR-28}, and the Trigger and Data Acquisition (TDAQ) system \cite{ATLAS-TDR-29}. 

%%%%%%%%%%%%%%%%%%%%%%%%%%%%%%%%%%%%%%%%%%%%%%%%%%%%%%%%%%%%%%%%%%

%%% Local Variables: 
%%% TeX-master: "mythesis"
%%% End: 



\chapter{ATLAS Software and Computing}
\label{thirdchapter}
The data collected from the detector must be compared to a set of simulated data in order to interpret efficiencies and background processes. These data sets aim to mimic different physics processes such as the events produced by the collider beams, the evolution of the collision products within the detector and materials, and the detector's response. The preparation of simulation starts off with Monte Carlo (MC) simulation, which is a computational technique that uses random sampling to generate events. Given these events, the interactions within the detector and the detector's response are simulated. This reconstructed product is called an Analysis Object Data (AOD), which are then cleaned by compressing the data and cutting any unnecessary events or columns into a finalized product called Derived AOD (DAOD). The products produced at each step are then stored into a compressed binary file, called a ROOT file, and are validated using different software tools. The software framework that encompasses the tools needed to produce, validate, and analyze all of these types of samples is called Athena \cite{ATLAS_Athena_Zenodo_3932810}. The flow of this process, compared to the similar process followed by collision data, is shown in Figure \ref{fig:DataChain}. 

\begin{figure}[ht]
\centerline{\includegraphics[height=95mm]{ch3_images/datachain.png}}
\caption[Data Chain]{ATLAS data chain-processing for data and Monte Carlo simulation \cite{ATLAS_Catmore_2020_DataProcessingChain}.}
\label{fig:DataChain}
\end{figure}

The studies for this thesis use data at the analysis level. This chapter will introduce ATLAS Open Data \cite{ATLAS_OpenData_DAODPHYSLITE_2015_2016}, ROOT, and the data structures of TTree versus RNTuple. It will also provide a comparison between the application programming interface (API) for the TTree and RNTuple formats.

\section{ATLAS Open Data}\label{sec:opendata}
ATLAS Open Data is a publicly available dataset produced by the ATLAS collaboration. It's composed of MC simulations of particle collisions within the ATLAS detector and detector data measurements. The data used as inputs for the remainder of this study are MC simulations of top samples from Run 2 \cite{ATLAS_OpenData_DAODPHYSLITE_MC_2024}. They are simulated samples of single top quarks, matter-antimatter $t\overline{t}$ pairs, and W boson production in association with jets. Representative diagrams for these processes are shown in Figure \ref{fig: samples}.

\begin{figure}
\centerline{\includegraphics[width=.8\textwidth]{ch3_images/samples.png}}
\caption[Data Chain]{The representative diagrams of the processes used in this thesis.}
\label{fig: samples}
\end{figure}

The inputs are all provided in DAOD\_PHYSLITE format, which contains already-calibrated objects directly from an AOD or PHYS product \cite{ATLAS_Collaboration_2023_PHYSLITE}. Those objects include jets, electrons, muons, photons, taus and their properties, such as momentum, mass, charge, eta, and phi. Each event contains a number of physical objects that depends on the underlying process, resulting in a multidimensional dataset. A full description of the variables can be found in \cite{ATLAS_OpenData_PHYSLITE_Docs}.

\section{ROOT}
% history of ROOT
ROOT is a unified software package developed for processing, analyzing, visualizing and ultimately storing the massive high-energy physics datasets. Previously, high-energy experiments used FORTRAN-based libraries; however, an upgrade was needed to handle the scales and complexities of the data from the LHC \cite{Brun:2296392}. ROOT maintains an object-oriented structure, meaning it is organized around the data rather than the functions and logic. Its features include visualization tools such as histogramming, and statistical tools. ROOT can be used in C++ and Python languages. ROOT’s declarative analysis interface (RDataFrame \cite{ROOT_RDataFrame_class}) is used extensively in this thesis, both in Python and in C++.

\subsection{ROOT Compression Algorithms}
ROOT offers four different compression algorithms: \texttt{ZLIB}, \texttt{LZMA}, \texttt{LZ4}, and \texttt{ZSTD} \cite{Marcon:2024zsm}. Data compression allows users to store large files at reduced sizes without losing information from the original file. It can also increase data reading and writing speeds. There are generally two types of compression algorithms: lossless and lossy. Lossy algorithms can reduce file sizes with larger compression factors, but are irreversible processes, meaning that information is lost during compression. The four compression algorithms from ROOT are lossless algorithms, meaning they are reversible processes that reduce bits by eliminating statistical redundancy.  

There are advantages and disadvantages in each of the four algorithms. \texttt{LZ4} focuses on compression and decompression speed, yet offers smaller compression factors and thus larger file sizes. \texttt{LZMA} provides higher compression at the cost of significantly slower reading speeds. \texttt{ZLIB} is an older version of \texttt{ZSTD}. Both provide a balance between compression and reading speeds; however, \texttt{ZSTD} has been shown to perform better in all metrics in comparison to ZLIB \cite{BockelmanShadura_2021_ZstdLZ4}. The default compression algorithm used to generation RNTuple samples is \texttt{ZSTD}. In Chapter \ref{fourthchapter}, a performance study using RNTuples produced with \texttt{LZ4} is shown for comparison.

\subsection{TTree Data Structure}
ROOT provides a columnar-based data structure called TTree to efficiently store data. The columnar-based format allows users to access independent columns of data, such as event IDs or particle kinematics, versus accessing information per event. Internally, the physical data of each column are stored as Binary Large Objects (BLOBs), which are compressed blocks of serialized values. These BLOBs are stored in containers called TBaskets, each holding the data for a consecutive range of entries. Each column of data has a corresponding TBranch object that stores metadata. The metadata describes the column, such as the column's data type and serialization rules, and if the columns has subcomponents, which are represented by associated TLeaf objects. For I/O (Input/Output) efficiency, TTrees group consecutive entries into units called clusters. 

The TBranch also stores metadata containing the list of file offsets and sizes of all of its baskets. ROOT uses these offsets and sizes to implicitly address the TBaskets and ultimately, decompress it and return the corresponding column data to the user. Figure \ref{fig:TTreeDataStructure} shows a more detailed flowchart of the TTree data structure.
\begin{figure}[ht]
\centerline{\includegraphics[height=130mm, width=.65\textwidth]{ch3_images/TTreeDataStructure.png}}
\caption[TTree Data Structure]{Representation of the TTree Data Structure \cite{ROOT_TTree_v6-30}.}
\label{fig:TTreeDataStructure}
\end{figure}

\subsection{RNTuple Data Structure}
RNTuple is the new columnar data format that will be implemented at the start of the HL-LHC. Its design continues to be columnar based, as its predecessor TTree, but it now uses modern storage technologies for better performance characteristics in data compactness, scalability, and read and write speed. For this reason, RNTuple classes are backwards-incompatible to TTree both on the file format level and API level \cite{ROOT_RNTuple_BinaryFormatSpecification}. Its binary format version follows an \emph{epoch.major.minor.path} scheme, where \emph{epoch} indicates backward-incompatible changes, \emph{major} indicates forward-incompatible changes, \emph{minor} indicates new optional format features, and \emph{patch} indicates backported features from newer format versions. The remainder of this study uses the first public release of RNTuple 1.0.0.0.

RNTuple separates logical schema from physical layout explicitly. Unlike TTree, RNTuple has an \emph{anchor}, which is a top-level metadata object that describes the schema, columns, and how the physical data is stored. Each column has a corresponding \emph{field}, which is an RNTuple object that maps the column to its physical storage location. Each column's data is stored in fixed-size units called \emph{pages}. Pages are the containers that hold the actual serialized BLOBs. Pages are grouped into larger units called \emph{clusters}, which define the granularity at which metadata and I/O decisions are made. Each cluster contains an \emph{envelope}, which are metadata blocks describing the physical locations of all the pages in that cluster. Inside envelopes are \emph{RBLOB} keys, which are the lowest-level object that hold the physical offset and size of a single page or multiple pages of the same cluster. Using RBLOB keys, RNTuple can explicitly locate, read, and decompress pages back to the user. A flow chart of this process in comparison to TTree is shown in Figure \ref{fig:RNTupleFlow}

Overall, this structure allows for more efficient random-access of individual events in comparison to TTree. For TTree, metadata is spread-out and stored in each individual TBranch; therefore,  its access pattern is more tightly coupled to sequential iteration. For RNTuple, the separation of the logical schema and physical layout allows for explicit addressing of the data and improves data compactness. 

\begin{figure}[ht]
\centerline{\includegraphics[width=\textwidth,height=100mm]{ch3_images/TTreevsRNTupleDataStructures.png}}
\caption[TTree vs. RNTuple Mapping]{TTree versus RNTuple mapping systems.}
\label{fig:RNTupleFlow}
\end{figure}

\section{TTree vs. RNTuple API}
The first portion of this thesis aimed to test RNTuple API against TTree using ATLAS analysis workflows. Efforts were made to document the capabilities, usage and best practices of RNTuple during its development. The corresponding code repository, including the aforementioned documentation can be found in \cite{Rodriguez_2025_RNTupleWorkflows}. The sections below will provide examples of RNTuple's API in comparison to TTree's, using native C++ event loops, RDataFrame, and Uproot \cite{uproot}. 

\subsection{Native C++ Event Loops}
Due to the multidimensional nature of particle physics data, event loops are common algorithms used in data analysis workflows. It is a process that continuously iterates through the large datasets to apply specific analysis steps to each event. As seen in Figure \ref{fig:eventloop-ttree}, users must iterate through TTree in order to load branches and define an empty pointer object to store their entries.

The RNTuple interface uses smart pointers, which simulate a pointer while providing automatic memory management \cite{Smart_Pointer_Wikipedia}. This feature shortens the amount of code necessary to read and load data by a couple of lines. For example \texttt{RNTupleReader::Open} simultaneously loads the ROOT file and the RNTuple, as seen in Figure \ref{fig:eventloop-rnt}. The function \texttt{GetView} also simultaneously loads and stores a field.

\begin{figure}[ht]
\begin{subfigure}{\linewidth}
\centerline{\includegraphics[width=.8\textwidth,height=100mm]{ch3_images/ttree.png}}
\caption{Native C++ event loop using TTree.}
\label{fig:eventloop-ttree}
\end{subfigure}
\begin{subfigure}{\linewidth}
\centerline{\includegraphics[width=\textwidth,height=90mm]{ch3_images/rntuple.png}}
\caption{Native C++ event loop using RNTuple.}
\label{fig:eventloop-rnt}
\end{subfigure}
\caption{These scripts load a DAOD\_PHYSLITE file containing a TTree in (a) and an RNTuple in (b) to plot the distribution of electron transverse momenta. Transverse momentum is the momentum perpendicular of the colliding beams.}
\end{figure}

\subsection{RDataFrame in C++ and Python}
RDataFrame is a high-level interface for analysis of data stored in TTree, RNTuple, CSV, and other data formats. RDataFrame can be used via C++ or Python languages. Analysis done with RDataFrame will mostly remain unmodified with RNTuple, as shown in Figure \ref{fig:rdfpython}, with the exception of filtering. Due to RNTuple's internal data structure, sub fields such as "\texttt{AnalysisElectronsAuxDyn:pt}" are separated by their field, "\texttt{AnalysisElectronsAuxDyn}" by a colon, instead of a period. This slight change confuses the filtering function in RDataFrame, but can be bypassed by assigning an alias name. Figure \ref{fig:rdfc} provides an example of how to read multiple inputs and apply a filter using RDataFrame in C++.

\begin{figure}[ht]
\begin{subfigure}{\linewidth}
\centerline{\includegraphics[width = 0.8\linewidth]{ch3_images/ttree-rdf-python.png}}
\caption{Reading multiple TTree inputs.}
\end{subfigure}
\begin{subfigure}{\linewidth}
\centerline{\includegraphics[width=0.8\linewidth]{ch3_images/rnt-rdf-python.png}}
\caption{Reading multiple RNTuple inputs.}
\end{subfigure}
\caption[Reading Multiple Inputs in RDataFrame Python: TTree vs. RNTuple]{Examples of how to load multiple inputs into an RDataFrame in Python.}
\label{fig:rdfpython}
\end{figure}

\begin{figure}[ht]
\begin{subfigure}{\linewidth}
\centerline{\includegraphics[width = 0.8\linewidth]{ch3_images/ttree-rdf.png}}
\caption{Reading multiple TTree inputs.}
\end{subfigure}
\begin{subfigure}{\linewidth}
\centerline{\includegraphics[width=0.8\linewidth]{ch3_images/rnt-rdf.png}}
\caption{Reading multiple RNTuple inputs.}
\end{subfigure}
\caption[Applying an RDataFrame Filter in C++ Using Multiple Inputs]{Examples of how to load multiple inputs into an RDataFrame and create a new filtered dataframe in C++.}
\label{fig:rdfc}
\end{figure}

\subsection{Uproot}
Uproot is a Python library designed to extract and manage data from ROOT files. Analysis done with Uproot will also remain unmodified with RNTuple. Users can load RNTuples and convert them into dictionaries of arrays just as with TTrees. Also, Uproot's added function, \texttt{uproot.mkrntuple}, allows users to write RNTuple outputs. Loading multiple RNTuple inputs using uproot's functions \texttt{uproot.concatenate} and \texttt{uproot.iterate} also work with RNTuple. 




\chapter{RNTuple vs. TTree Performance}
\label{fourthchapter}
In this chapter, RNTuple performance is analyzed using RDataFrame and compared to TTree. First, 92 TTrees stored in \texttt{DAOD\_PHYSLITE} files from ATLAS Open Data were converted to RNTuples using its default compression algorithm setting, ZSTD. An average size reduction of about 47\% was observed between the converted RNTuples and the original TTrees, as shown in Figure \ref{fig:conversionsize}. Speed tests were performed for loading and outputting RNTuples in comparison to TTrees using \texttt{std::chrono::high\_resolution\_clock::now()}. Each performance study contains two versions: a TTree version that uses TTree inputs and an RNTuple version that uses RNTuple inputs. A comparison of peak memory consumption was also performed using both sets of inputs. The entirety of this analysis was repeated for RNTuple inputs that were converted with LZ4 compression algorithm. 
\begin{figure}[H]
\centerline{\includegraphics[height=95mm]{ch4_images/RNT_TTree_size_comparisons.png}}
\caption[TTree vs. RNTuple Sizes]{The RNTuple:TTree file size ratios over number of events per file.}
\label{fig:conversionsize}
\end{figure}

\section{Readability Speed}
The total loading times for 92 RNTuples and their TTree equivalence were measured 100 times to ensure consistency. Loading multiple RNTuples in RDataFrame follows an identical procedure in both TTree and RNTuple versions (seen previously in 3.3.2). The timer began at the start of the script and was stopped after calculating the sum of the column "\texttt{AnalysisElectronsAuxDyn:pt}". This was done to ensure that the data was being loaded and read by RDataFrame. The measured times were recorded onto a text file and are shown in Figure \ref{fig:loading}. In comparison to TTree, this study finds RNTuple to be 2.38 times faster at loading a column of data. 
\begin{figure}[H]
\centerline{\includegraphics[height=95mm]{ch4_images/LoadingHistograms.png}}
\caption[Distribution of Total Loading Times]{Total loading times measured for TTree and RNTuple using RDataFrame in C++.}
\label{fig:loading}
\end{figure}

\section{Writing Speed}\label{sec:writingspeed}
\label{sec:writing}
Writing speed was measured by performing an invariant mass calculation and outputting a new dataset with two columns: "\texttt{ElectronPairsInvMass}" and "\texttt{MuonPairsInvMass}". The timer began at the start of an invariant mass calculation and stopped after creating a new dataset. A TTree was written for the TTree version and an RNTuple was written for the RNTuple version. The quick function that outputs a TTree in RDataFrame, \texttt{df.Snapshot(...)}, is currently not developed to output an RNTuple yet; therefore, for consistency, both versions of the script uses the RDataFrame function \texttt{df.ForEach(...)} to loop through events and fill in the new columns. This procedure for RNTuple and TTree versions is shown in Figure \ref{fig:writing-procedure}. 

\begin{figure}[h!]
\begin{subfigure}{\linewidth}
\centerline{\includegraphics[width = 0.8\linewidth]{ch4_images/ttree-writing.png}}
\caption{TTree Version.}
\end{subfigure}
\begin{subfigure}{\linewidth}
\centerline{\includegraphics[width=0.8\linewidth]{ch4_images/writing_rnt.png}}
\caption{RNTuple Version.}
\end{subfigure}
\caption[Writing a Two Column Output Algorithm Using RDataFrame in C++]{TTree vs. RNTuple writing algorithms using the RDataFrame function \texttt{df.ForEach(...)} in C++.}
\label{fig:writing-procedure}
\end{figure}

Although the procedures are the same, the RNTuple version takes up significantly less code due to the RNTuple API. With TTree, an empty vector has to be created before writing a branch. With RNTuple, the \texttt{ROOT::RNTupleModel} class has the function \texttt{MakeField}, which creates a new field given a name and a corresponding value managed by a shared pointer. The function \texttt{RNTupleWriter::Recreate()} simultaneously creates the RNTuple and the output ROOT file. In the TTree version, both the TTree and the output file need to be defined separately. 

The total output times were measured 100 times and  were recorded in a text file. The results in Figure \ref{fig:writing} show that writing with RNTuple is 1.51 times faster than with TTrees.  
\begin{figure}[h!]
\centerline{\includegraphics[height=95mm]{ch4_images/OutputHistograms.png}}
\caption[Distribution of Total Writing Times]{Total writing times measured for TTree and RNTuple using RDataFrame in C++.}
\label{fig:writing}
\end{figure}

\section{Output Sizes}
The output file sizes were measured to determine whether RNTuple maintains a consistent size-reduction behavior at this stage of the analysis. By error, the outputs initially produced contained empty events; however, this brought some insights on RNTuple when compared to "cleaned" outputs that filtered out empty events. The results shown in Table \ref{table:dirty}, reveal that RNTuple provides a 99\% event size reduction to TTree when the outputs written include empty events. This implies that RNTuple is handling repeated bits significantly better than TTree. Table \ref{table:clean} reveals a 63\% reduction from RNTuple when eliminating the empty events. The latter result is considered more practical or realistic for an analysis; yet, these results open an opportunity to write data and approach analysis workflows differently. 
\begin{table}[htb]
\caption[File Size and Avg. Event Size of UnFiltered Output]{\label{table:dirty}
File size and averaged compressed event size for TTree and RNTuple outputs with empty events. The total number of unfiltered events written is 9,045,000 events.}
\begin{center}
\begin{tabular}{ m{4cm} m{4cm} m{6cm} }
\hline
DataFormat & File Size [bytes] & Average Compressed Event Size [bytes/event] \\
\hline
TTree & 48 086 740 & ~ 5.31 \\
RNTuple & 447 414 & ~ 0.049 \\
\hline
\end{tabular}
\end{center}
\end{table}

\begin{table}[htb]
\caption[File Size and Avg. Event Size of Filtered Output]{\label{table:clean}
File size and averaged compressed event size for TTree and RNTuple outputs without empty events. The total number of filtered events is 77,411 events.}
\begin{center}
\begin{tabular}{ m{4cm} m{4cm} m{6cm} }
\hline
DataFormat & File Size [bytes] & Average Compressed Event Size [bytes/event] \\
\hline
TTree & 791 428 & ~ 10.23 \\
RNTuple & 288 529 & ~ 3.73 \\
\hline
\end{tabular}
\end{center}
\end{table}

\section{Memory Consumption}
Peak memory usage was measured using Python versions of the writing scripts used in \ref{sec:writingspeed}. Using the command \texttt{usr/bin/time}, memory usage was measured 100 times for both TTree and RNTuple versions. For this test study, results shown in Figure \ref{fig:peak-memory} demonstrate no significant difference between RNTuple and TTree.
\begin{figure}[ht]
\centerline{\includegraphics[height=95mm]{ch4_images/memory_bargraph.png}}
\caption[Peak Memory Usage: Writing Two Column Output]{Peak memory usage while producing an output with two columns. Measurements were taken 100 times for each version.}
\label{fig:peak-memory}
\end{figure}

\section{LZ4 Compression Algorithm Study}
Studies have shown that LZ4 improves reading and writing speeds for TTree, but at the cost of larger files \cite{Marcon:2024zsm}. This section will investigate if this behavior is consistent with RNTuple by repeating the loading and writing measurements. The same 92 ATLAS Open Data files were used to produce RNTuple equivalents with the LZ4 compression algorithm specified. Time measurements for loading the electron transverse momenta column are shown in Figure \ref{fig:lz4-loading}, and the time measurements for writing an RNTuple output are shown in Figure \ref{fig:lz4-writing}. There are no significant differences between reading RNTuples produces from LZ4 or ZSTD algorithms; however, there is a 2 second difference between writing an RNTuple with LZ4 versus ZSTD. The ratio of the LZ4 RNTuple sizes over the RNTuples produced with ZSTD are shown in Figure \ref{fig:lz4-sizes}. They reveal that the LZ4 algorithm increases the RNTuple file sizes by an average of 14\% from ZSTD. 

\begin{figure}[ht]
\centerline{\includegraphics[height=95mm]{ch4_images/LoadingHistograms_rntuplesLZ4.png}}
\caption[Distribution of Loading Times Using LZ4 Inputs]{Loading time measurements for RNTuples produced by the LZ4 and ZSTD algorithms, and for TTree. The RNTuples composed with ZSTD and LZ4 only differ by a couple of milliseconds.}
\label{fig:lz4-loading}
\end{figure}

\begin{figure}[ht]
\centerline{\includegraphics[height=95mm]{ch4_images/OutputHistograms_LZ4.png}}
\caption[Distribution of Writing Times Using LZ4 Inputs]{Writing time measurements for RNTuples produced by the LZ4 and ZSTD algorithms, and TTree.}
\label{fig:lz4-writing}
\end{figure}

\begin{figure}[ht]
\centerline{\includegraphics[height=95mm]{ch4_images/LZ4_size_comparisons.png}}
\caption[Per-file Compression Ratios of LZ4:ZSTD Over Total Number of Events]{Per-file compression ratios of LZ4:ZSTD over total number of events.}
\label{fig:lz4-sizes}
\end{figure}

\section{Performance Discussion}
Common analysis steps used in RDataFrame workflows have improved with RNTuple compared to its TTree predecessor. Converting TTree to RNTuple showed immediate reductions in file size on disk. Reading and writing speeds increased without any significant cost to memory usage. Additionally, the RNTuple API reduces lines of code, making it more user-friendly than the TTree API. 

The comparisons between RNTuples produced with LZ4 versus ZSTD provide a first look at how RNTuple behaves with different compression algorithms. RNTuples produced with LZ4 show improved reading and writing speeds, though not at significant levels and at the cost of increased disk size. Given this study, producing RNTuples with ZSTD is recommended. 

\chapter{Analysis Grand Challenge: RNTuple vs. TTree}
\label{fifthchapter}
The Analysis Grand Challenge (AGC) is an analysis on top quark production meant to showcase an end-to-end analysis pipeline. Developed and organized by Iris-HEP, the AGC has several versions that showcase different cyber infrastructure and workflows, making it a great benchmark to test RNTuple. This section will describe the development of two new AGC versions that use ATLAS Open Data and RDataFrame: TTree and RNTuple versions. These versions were heavily influenced on the existing RDataFrame AGC repository that applies CMS open data and the uproot AGC repository that uses ATLAS Open Data. The full implementations of the AGC can be found in \cite{Rodriguez_2025_RNTupleWorkflow}. 

\section{RDataFrame Analysis Workflow}
The AGC is divided into two parts: an analysis script and a statistical script, both written in Python. The analysis scripts uses RDataFrame to apply preselections and output histograms of the top quark mass and the scalar sum of the transverse momenta, $H_T$, into a ROOT file. The statistical script performs a simple statistical analysis using the output ROOT file from the analysis script.

The inputs used for the AGC are the same 92 ROOT file from ATLAS Open Data, as described in \ref{sec:opendata}. Specifically, there are 22 single top samples, 10 $t\overline{t}$ samples, and 60 W+jets samples.

\subsection{Event Selections}
To reconstruct the top quark mass, events are selected from top quark pair production with final states that include a single charged lepton corresponding to the signature of semileptonic $t\overline{t}$ events, as shown in Figure \ref{fig:ttbar}. The leptons must have $p_t$ larger than 30 GeV and $|\eta|$ less than 2.1 events must include four jets, with two of the four being b-tagged. The other two jets are from the W boson decay. The top mass observable is then reconstructed by taking the invariant mass of the trijet with the largest transverse momentum, $p_t$. To plot the $H_T$ observable, the selected events must have at least one b-tagged jet among the four jets and exactly one lepton.
\begin{figure}[ht]
\centerline{\includegraphics[height=95mm]{ch6_images/ttbar.png}}
\caption[Top and Anti-top Quark Collision]{The schematic view of a top and anti-top quark collision \cite{Held_2022_PyHEP2022_AGC_talk}}
\label{fig:ttbar}
\end{figure}

The results of the newly developed AGC using ATLAS Open Data are shown in Figures \ref{fig:top-mass} and \ref{fig:Ht}. Both the RNTuple and TTree versions produced the same output, confirming that analysis performed in RDataFrame using RNTuple will remain largely unmodified. As previously mentioned, RNTuple only changes the structure of variable field names; therefore, alias variable names were applied to both TTree and RNTuple versions for consistency. 
\begin{figure}[ht]
\centerline{\includegraphics[height=95mm]{ch6_images/trijet_mass_prefit-2.pdf}}
\caption[The Trijet Mass Prefit]{The trijet mass prefit. This result is the same for both RNTuple and TTree versions of the AGC.}
\label{fig:top-mass}
\end{figure}

\begin{figure}[ht]
\centerline{\includegraphics[height=95mm]{ch6_images/Ht_prefit-3.pdf}}
\caption[Ht]{The $H_T$ observable prefit. This result is the same for both RNTuple and TTree versions of the AGC.}
\label{fig:Ht}
\end{figure}

\section{AGC Performance Studies}
A performance study evaluating execution speed and memory usage was conducted for both TTree and RNTuple versions of the AGC. Total execution times were measured 100 times for each version using the Python \emph{time} library. Both versions used inputs produced with the ZSTD compression algorithm. As shown in Figure \ref{fig:AGC-time}, RNTuple averaged 47.58 seconds to produce the top quark mass and $H_T$ histograms into a ROOT file, while TTree averaged 71.75 seconds. RNTuple was approximately 1.51 times faster, consistent with previous time measurements shown in Chapter \ref{fourthchapter}. The execution times were then remeasured using RNTuples produced with the LZ4 compression algorithm. As shown in Figure \ref{fig:AGC-LZ4}, LZ4 yields a slight improvement on the order of a few seconds, which is also consistent with previous results. Peak memory usage was measured using the inputs produces with ZSTD and with \texttt{/usr/bin/time}. As shown in Figure \ref{fig:memory-AGC}, RNTuple consumes slightly less memory usage than TTree when executing the AGC analysis script.
\begin{figure}[ht]
\centerline{\includegraphics[height=95mm]{ch6_images/TimeDistribution.png}}
\caption[Distribution of AGC Execution Times]{The total execution times of the AGC measured 100 times for TTree and RNTuple versions.}
\label{fig:AGC-time}
\end{figure}

\begin{figure}[ht]
\centerline{\includegraphics[height=95mm]{ch6_images/TimeDistribution_LZ4.png}}
\caption[Distribution of AGC Execution Times using LZ4 Inputs]{The total execution times of the AGC measured 100 times with RNTuples produced with the LZ4 compression algorithm.}
\label{fig:AGC-LZ4}
\end{figure}

\begin{figure}[ht]
\centerline{\includegraphics[height=95mm]{ch6_images/memory_histogram.png}}
\caption[AGC Peak Memory Usage]{The peak memory usage when executing the AGC.}
\label{fig:memory-AGC}
\end{figure}





\chapter{Conclusion}
\label{conclusion}
\chapter{Analysis Grand Challenge: RNTuple Implementation}		
\label{sixthchapter}

The Analysis Grand Challenge (AGC) is an analysis on t quark production meant to showcase an end-to-end analysis pipeline \cite{AGC}. Developed and organized by Iris-HEP \cite{Iris-HEP}, the AGC has several versions that showcase different cyber infrastructure and workflows, making it a great benchmark to test RNTuple. This section will describe the development of two new AGC versions that use ATLAS Open Data and RDataFrame: TTree and RNTuple versions. These versions were heavily influenced on the existing RDataFrame AGC repository that applies CMS open data and the uproot AGC repository that applies ATLAS open data \cite{AGC}.  

\section{RDataFrame Analysis Workflow}
The AGC is divided into two parts: an analysis script and a statistical script. The analysis scripts are written in Python and uses RDataFrame to apply preselections and output histograms of the top quark mass and the scalr sum of the transverse momenta,$H_T$, into a root file. The statistical script performs a simple statistical analysis using the output root file from the analysis script.

\subsection{Event Selections}
To reconstruct the top quark mass, events are selected from top quark pair production with final states that include a single charged lepton, as shown in Figure \ref{fig:ttbar}. The leptons must have $p_t$ larger than 30 GeV and $|\eta|$ less than 2.1 Events must include four jets, with two of the four being "b-tagged". Jets that are "b-tagged" are matched to $b$ and $\overline{b}$. The other two jets are from the W boson decay. The top mass observable is then reconstructed by taking the invariant mass of the trijet with the largest transverse momentum,$p_t$. The results are shown in Figure \ref{fig:top-mass}.
\begin{figure}
\centerline{\includegraphics[height=95mm]{ch6_images/ttbar.png}}
\caption[ttbar]{The schematic view of a top and anti-top quark collision \cite{Alex}}
\label{fig:ttbar}
\end{figure}

\begin{figure}
\centerline{\includegraphics[height=95mm]{ch6_images/trijet_mass_prefit-2.pdf}}
\caption[top mass]{The trijet mass prefit. This result is the same for both RNTuple and TTree versions of the AGC.}
\label{fig:top-mass}
\end{figure}

To plot the $H_T$ observable, the selected events must have at least one b-tagged jet among the four jets and exactly one lepton. The results are shown in Figure \ref{fig:Ht}.
\begin{figure}
\centerline{\includegraphics[height=95mm]{ch6_images/Ht_prefit-3.pdf}}
\caption[Ht]{The $H_T$ observable prefit. This result is the same for both RNTuple and TTree versions of the AGC.}
\label{fig:Ht}
\end{figure}

\section{TTree vs. RNTuple AGC}
Both TTree and RNTuple versions of the AGC produced the outputs, confirming that analysis done in RDataFrame with RNTuple will mostly remain unmodified. As previously mentioned, RNTuple changes the structure of variable field names. For consistency, alias variable names were applied to both TTree and RNTuple versions. 

\subsection{Timing Measurements}
Total execution times were measured 100 times for both TTree and RNTuple versions using the time python library. Both versions used inputs produced with ZSTD compresseion algorithm. The results, shown in Figure \ref{fig:AGC-time}, show that RNTuple averaged 47.58 seconds to produce the top quark mass and $H_T$ histograms into a root file, while TTree averaged 71.75 seconds. RNTuple was about 1.51 times faster, which is consistent with previous time measurements shown in Chapter \ref{fourthchapter}
\begin{figure}
\centerline{\includegraphics[height=95mm]{ch6_images/TimeDistribution.png}}
\caption[Ht]{The total execution times of the AGC measured 100 times for TTree and RNTuple versions.}
\label{fig:AGC-time}
\end{figure}

\subsubsection{LZ4 vs. ZSTD Input Files}
The total execution times were remeasured using RNTuple inputs produced with the LZ4 compression algorith. As shown in Figure \ref{fig:AGC-LZ4}, LZ4 executes the analysis script of the AGC about a couple seconds faster. 
\begin{figure}
\centerline{\includegraphics[height=95mm]{ch6_images/TimeDistribution_LZ4.png}}
\caption[LZ4]{The total execution times of the AGC measured 100 times with RNTuples produced with LZ4 compression algorith.}
\label{fig:AGC-time}
\end{figure}

\subsection{Memory Consumption}
Peak memory usage was also measured using \texttt{/usr/bin/time}. The results shown in Figure \ref{memory-AGC}, show that RNTuple consumes slightly less memory usage when executing the analysis script than TTree. 
\begin{figure}
\centerline{\includegraphics[height=95mm]{ch6_images/memory_histogram.png}}
\caption[LZ4]{The peak memory usage when executing the AGC.}
\label{fig:memory-AGC}
\end{figure}


\begingroup
\setstretch{1}
\setlength\bibitemsep{20pt}
\addcontentsline{main.toc}{chapter}{REFERENCES}
\printbibliography[
    heading=bibintoc,
    title={REFERENCES}
]
\endgroup
\end{document}
