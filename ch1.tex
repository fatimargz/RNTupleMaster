\chapter{Introduction}		% chapter 1
\label{introchap}		% for reference (\ref{introchap})
% what is particle physics and why is it important to the world? 
Our current understanding of the building blocks of our universe is summarized with one model, called the Standard Model (SM). From the way we power our cities, to the particles that hold them together, the SM explains how the basic building blocks of matter interact, governed by the four fundamental forces: gravity, electromagnetism, the strong force and the weak force. Yet, questions remain about the SM, such as why are there only three generations of fundamental particles? What is the nature of dark matter and dark energy, and how does it fit within the SM? What about the origin of the matter-antimatter asymmetry? Is there a unification theory for the fundamental forces? Is the SM complete or do other exotic particles exists? Over the years, experimental particle physicists and engineers have built technology to test the SM, either by performing precision measurements of particles and their behaviors, or by colliding particles and measuring their outputs. As a result, we have increased our confidence in the SM theory, but continue to search for answers for these remaining questions through experimental discovery.  
% add SM figure
\begin{figure}
\begin{subfigure}{0.5\textwidth}
\includegraphics[width=0.9\linewidth, height=6cm]{ch1_images/SMSummary.png}
\end{subfigure}
\begin{subfigure}{0.5\textwidth}
\includegraphics[width=0.9\linewidth,height=6cm]{ch1_images/SMSummary_b.png}
\end{subfigure}
\caption[Summary of SM cross-section Measurments]{
	Summary of several Standard Model cross-section measurements (a) with associated references (b) \cite{ATLASSummaryPlots}. Cross-sections can be thought of as probabilities that a process occurs. This means that the processes with smaller cross-sections are considered rare-processes because it has a lower probability of being observed. Increasing the probability of these rare-processes would require an increase of luminosity or collisions. The measurements are corrected for branching fractions, compared to the corresponding theoretical expectations. 
	}
\label{fig:SM_cross-sections}
\end{figure}

A Toroidal LHC Apparatus (ATLAS) is a particle physics experiment designed to detect the high-energy particle collisions from the Large Hadron Collider (LHC). At the LHC, collisions take place at a rate of more than a billion interactions per second, which is a combined data volume of about 60 million megabytes per second. However, in order to study rare processes, as shown in Figure \ref{fig:SM_cross-sections}, the LHC will have a major upgrade to increase the number of collisions by a factor of 5 to 7.5. This upgrade, called the High-Luminosity LHC, will require a new data storage format that can handle this increase in data.
% I need to edit the wording in this paragraph

%%%Introduce what RNTuple is and why it will be replacing TTree.
RNTuple is the new ROOT data storage format that will be in use at the start of the HL-LHC. Due to its design, which takes advantage of modern C++ techniques, it is set to improve read speedability and memory usage compared to its predecessor, TTree, and other data storage formats such as HDF5 and Parquet. At the start of this work, performance studies on RNTuple were conducted at the production level, and RNTuple was still at an experimental stage. 

%%%Paragraph on paper outline.
This thesis investigates the performance of RNTuple for ATLAS analysis workflows. This chapter will provide a more detail introduction of the SM, followed by an introduction to the ATLAS experiment and its detector technology in Chapter 2. In Chapter 3, the ATLAS software and computing system, and data contents are introduced. In Chapter 4, an introduction to RNTuple and TTree is provided along with examples of how RNTuple is applied in comparison to TTree. Performance studies conducted for RNTuple and how they compare with TTree will be presented in Chapter 5. In Chapter 6, the Analysis Grand Challenge (AGC) is introduced along with its RNTuple implementation. A final discussion and conclusions are given in Chapter 7.

\subsection{Standard model of particle physics}
% one paragraph providing an overview of particle characteristics for matter vs. forces., units, bosons vs fermions. 
The SM is a quantum field theory that explains and catagorizes all observed fundamental particles by their properties and interactions. Quantum field theory (QFT) is the main theoretical tool for describing particle interactions by combining special relativity and quantum mechanics. Due to this combination, QFT is a probabilistic theory where each particle has an associated field that permeates all of space; therefore, forces are simply the interactions between these different fields. For example, the electromagnetic force is just the interaction between the electromagnetic field and charged matter fields, which fall under quantum electrodynamics (QED). In sum, the SM encompasses all known elementary particle interactions, except for gravity, through a collection of quantum field theories, each dictated by gauge symmetries: QED ($U(1)$), the Glashow-Weinberg-Salam theory of electroweak processes ($SU(3)$), and quantum chromodynamics ($SU(2) x U(1)$).

\subsubsection{Symmetries and Particle Content}
In physics, symmetries are fundamental because they lead to conservation laws through Noether's Theorem. Symmetries can manifest in two notions: invariance and covariance. Properties of a system are described as invariant if they do not change under a symmetry transformation. For example, rotating a sphere and without altering gravitational force would indicate a conservation of angular momentum. In contrast, covariance is used to describe a system that changes in accordance to changes induced by symmetry transformations. 

The SM is a gauge theory based on the symmetry group $SU(3)_C x SU(2)_L x U(1)_Y$. Gauge theory is a QFT that requires invariance under continous transformations, and a symmetry group is a set objects that obey the four properties listed in Table \ref{group_axioms}. $SU(3)_c$ is called the color symmetry group describing the strong nuclear force, which is the intraction between quarks and gluons. $SU(2)_L x U(1)_Y$ describes the electromagnetic and weak nuclear forces, which is the interactions between leptons, photons and W\/Z bosons. 
% axioms of a group from Robert Mann book
\begin{table}[htb]
\caption[Properties of a Group.]{\label{group_axioms}
Properties of a Group \cite{Mann:2010nvj}.
}
\begin{center}
\begin{tabular}{c c c}
\hline
CLOSURE & If $g_1, g_2 \in G \rightarrow g_1 \diamond g_2 \in G$ & (combinations remain in the set) \\
IDENTITY & There exists $I \in G \rightarrow I \diamond g_i = g_i$ for every $g_i \in G$ & (one element does nothing) \\
INVERSE & Every $g_i \in G$ has a $g_i^{-1} \in G$ such that $g_1 \diamond g_i^{-1} = I$ & (combinations can be undone) \\
ASSOCIATIVITY & If $g_1, g_2, g_3 \in G \rightarrow (g_1 \diamond g_2) \diamond g_3 = g_1 \diamond (g_2 \diamond g_3)$ & (combinational groupings can be interchanged) \\ 
\hline
\end{tabular}
\end{center}
\end{table}

Furthermore, the four groups of particles shown in Figure \ref{fig:SM_wikid}: quarks, leptons, gauge bosons, and scalar bosons, can be further categorized as \emph{bosons} or \emph{fermions} because of a fundamental property called spin. Similar to the Earth, particles carry orbital angular momentum and spin angular momentum; however, for particles, spin is an intrinsic property. All bosons carry an integer spin; meanwhile, fermions carry half-integer spin. As a result from QFT, each fermion has an antiparticle with the same mass and lifetime as the particle itself, but oppositely charged. The three charged leptons ($e$, $\mu$, $\tau$) are massive, while their corresponding nuetrinos ($\nu_e$,$\nu_{\mu}$,$\nu_{\tau}$), are massless with nuetral charge. Due to QCD, there are 8 types of gluons. The Higgs boson has its own section as a scalar boson because unlike the vector bosons with spin 1, the Higgs boson has spin 0. In sum, there are a total of 12 leptons including their antiparticles, 36 quarks including all the flavors and their antiparticles, 12 vector bosons, and 1 scalar boson, which makes a total of 61 fundamental particles.
% Standard Model Figure
\begin{figure}
\centerline{\includegraphics[height=95mm]{ch1_images/Standard_Model_of_Elementary_Particles.png}}
\caption[The SM.]{Particle content of the Standard Model \cite{SM-wikidpedia}.}
\label{fig:SM_wikid}
\end{figure}

\subsection{Standard Model Limitations}



\subsection{Phenomenology of Large Hadron Colliders}


%%% Local Variables: 
%%% TeX-master: "mythesis"
%%% End: 
