Our current understanding of the building blocks of our universe is summarized with one model, called the Standard Model (SM)  \cite{Mann:2010nvj}. From the way we power our cities, to the particles that hold them together, the SM explains how the basic building blocks of matter interact, governed by fundamental forces: electromagnetism, the strong force and the weak force. Yet, questions remain about the SM, such as is there a unification theory that includes gravity? Why are there only three generations of fundamental particles? What is the nature of dark matter and dark energy, and how does it fit within the SM? What about the origin of the matter-antimatter asymmetry?  Is the SM complete or do other exotic particles exists? Over the years, experimental particle physicists and engineers have built technology to test the SM, either by performing precision measurements of particles and their behaviors, or by colliding particles and measuring their outputs. As a result, we have increased our confidence in the SM theory, but continue to search for answers for these remaining questions through experimental discovery.  

A Toroidal LHC Apparatus (ATLAS) \cite{CERN:1992jvx} is a particle physics experiment designed to detect the high-energy particle collisions from the Large Hadron Collider (LHC) \cite{Bruning:2004ej}. At the LHC, collisions take place at a rate of more than a billion interactions per second, which is a combined data volume of about 60 million megabytes per second \cite{ATLAS_TriggerDAQ}. In order to extend its discovery potential, the LHC will have a major upgrade to increase the number of instantaneous collision rate. This upgrade, called the High-Luminosity LHC (HL-LHC) \cite{ZurbanoFernandez:2020cco}, will require a new data storage format that can handle this increase in data.

RNTuple \cite{refId0} is the new ROOT \cite{ROOT_Documentation} data storage format that will be in use at the start of the HL-LHC \cite{Blomer_2020}. RNTuple takes advantage of modern C++ techniques, which have shown to improve read speedability and memory usage when compared to its predecessor, TTree, and other data storage formats such as HDF5 and Parquet \cite{Lopez_Gomez_2023}. RNTuple is curently under heavy development. Its base format has only recently left the experimental stage and many tools and capabilities built around it are still evolving.

This thesis investigates the performance of RNTuple for ATLAS analysis workflows. This chapter will provide a more detailed introduction of the SM, followed by an introduction to the ATLAS experiment and its detector technology in Chapter 2. In Chapter 3, the ATLAS software and computing system, and data contents are introduced. In Chapter 4, an introduction to RNTuple and TTree is provided along with examples of how RNTuple is applied in comparison to TTree. Performance studies conducted for RNTuple and how they compare with TTree will be presented in Chapter 5. In Chapter 6, the Analysis Grand Challenge (AGC) \cite{IRISHEP_AGC} is introduced along with its RNTuple implementation. A final discussion and conclusions are given in Chapter 7.

\section{Phenomenology at the LHC}
The SM is a quantum field theory that explains and catagorizes all observed fundamental particles by their properties and interactions. Quantum field theory (QFT) is the main theoretical tool for describing particle interactions by combining special relativity and quantum mechanics. Due to this combination, QFT is a probabilistic theory where each particle has an associated field that permeates all of space; therefore, forces are simply the interactions between these different fields. For example, the electromagnetic force is just the interaction between the electromagnetic field and charged matter fields, which fall under quantum electrodynamics (QED). In sum, the SM encompasses all known elementary particle interactions, except for gravity, through a collection of quantum field theories: QED, the Glashow-Weinberg-Salam theory of electroweak processes, and quantum chromodynamics.

The four groups of particles shown in Figure \ref{fig:SM_wikid}: quarks, leptons, gauge bosons, and scalar bosons, can be further categorized as \emph{bosons} or \emph{fermions} because of a fundamental property called spin. Similar to the Earth, particles carry orbital angular momentum and spin angular momentum; however, for particles, spin is an intrinsic property. All bosons carry an integer spin; while, fermions carry half-integer spin. As a result from QFT, each fermion has an antiparticle with the same mass and lifetime as the particle itself, but are oppositely charged. The three charged leptons ($e$, $\mu$, $\tau$) are massive, while their corresponding nuetrinos ($\nu_e$, $\nu_{\mu}$, $\nu_{\tau}$), are massless with nuetral charge. Due to QCD, there are six flavors or types of quarks (up, down, strange, charm, top, and bottom), each of which carries an intrinsic property called color charge. There are three values of color charge (red, green and blue) resulting to 8 distinct gluons. The Higgs boson has its own section as a scalar boson because unlike the vector bosons with spin 1, the Higgs boson has spin 0. In sum, there are a total of 12 leptons including their antiparticles, 36 quarks, also including their antiparticles, 12 vector bosons, and 1 scalar boson, which makes a total of 61 fundamental particles.
% Standard Model Figure
\begin{figure}
\centerline{\includegraphics[height=95mm]{ch1_images/Standard_Model_of_Elementary_Particles.png}}
\caption[The Standard Model]{Particle content of the Standard Model \cite{ElementaryParticle_Wikipedia}.}
\label{fig:SM_wikid}
\end{figure}

Collider experiments serve as probes to the SM because they directly test conservation laws through the detection of final state radiation. In colliders, two beams of particles are accelerated to reach high energies and brought together for collision. Each collision is called an event and specific interactions or transformations are called processes. Processes are governed by conservation laws, such as conservation of energy and charge. For example, due to the conservation of energy, the energy in the center of mass frame must be greater than the sum of masses of the produced particles. When high-energy charged particles pass through matter, they ionize atoms along their path, which then serve as "seeds" for cloud chambers or sparks for sparks chambers. Their classification is then calculated by the energy differences detected from those "seeds". For neutral particles, their reconstruction is calculated using the conservation of momentum. Through QFT, the rate of a process, called cross-sections, can be predicted via the particle kinematics, their properties, and the properties of the process. Experimentally, cross-sections can be calculated via Equation \ref{eq:cross-section}, where $N$ is the number of events for the process being measured and $L$ is the instantaneous luminosity, defined in Equation as \ref{eq:luminosity}. 
\begin{equation}
	\sigma = \frac{N}{\int L dt}
\label{eq:cross-section}
\end{equation}
\begin{equation}
	L = f \frac{n_1 n_2}{4\pi\sigma_x\sigma_y}
\label{eq:luminosity}
\end{equation}
$f$ is the frequency of collisions, $n_1$ and $n_2$ are the number of particles in the colliding bunches. $\sigma_x$ and $\sigma_y$ are the root-mean-squared horizontal and vertical beam sizes. Figure \ref{fig:SM_cross-sections} displays the predicted cross-sections for certain processes and the required center of mass energies for those processes to be observed. Processes with smaller cross-sections are considered rare-processes because they have a lower probability of being observed; therefore, increasing the probability of these rare-processes would require an increase of energy.  
\begin{figure}
\centerline{\includegraphics[height=95mm]{ch1_images/SMSummary.png}}
\caption[Summary of SM Cross-section Measurments]{
	Summary of several Standard Model cross-section measurements. The associated references can be found in Reference \cite{ATLAS:2024cgh}. The measurements are corrected for branching fractions, compared to the corresponding theoretical expectations. 
	}
\label{fig:SM_cross-sections}
\end{figure}

\section{Physics Quantities}
This section will cover some relavant physics quantities. 
\subsection{Invariant Mass}
Invariant mass is a quantity that characterizes a system's total energy and momentum independent of the overall motion of the system \cite{ATLAS_MassGlossary}. Due to special relativity, space and time coordinates are linked, but dependent on a frame of reference. Lorentz transformations are used to convert coordinates from one reference frame to another, and four-vectors are used to simplify these transformations \cite{VanWijk_4VectorsInvariantMass}. A four-vector represents a physical quantity in space-time. For example, the position four-vector includes the spatial coordinates (x, y, z) and time, while the four-momentum vector includes the energy and the momentum coordinates in the x, y, and z directions. Four-vectors provide a convenient framework for calculating invariant quantities such as the invariant mass of a resonance that has decayed into other particles. 

The invariant mass of oppositely charged muons and electrons is calculated for studies in Chapter \ref{fifthchapter}. A lepton selection with transverse momentum greater than 25 GeV is applied first to suppress background, followed by the pairing of oppositely charged leptons. Their invariant mass is then calculated using Equation \ref{eq:invariant_mass}, where $p_x$, $p_y$, $p_z$ is momentum in the x, y, z directions and $E$ is energy. The peak of the invariant mass distribution, Figure \ref{fig: invariant_mass}, returns the Z boson mass at 75.1338 GeV.
\begin{equation}
m = \sqrt{\sum E^2 - \sum p_x^2 - \sum p_y^2 - \sum p_z^2}
\label{eq:invariant_mass}
\end{equation}

\begin{figure}
\centerline{\includegraphics[height=95mm]{ch1_images/LeptonPairInvm_RDFTTree.png}}
\caption[Invariant Mass Distribution of Oppositely Charged Lepton Pairs.]{Invariant Mass distribution of oppositely charged lepton pairs using data highlighted in Chapter \ref{thirdchapter}.}
\label{fig: invariant_mass}
\end{figure}

\subsection{Jets}
Jets are energy deposits in the detector that are grouped together to represent quarks and gluons, collectively known as partons \cite{ATLAS_OpenData_JetsDoc}. Partons cannot be observed in isolation due to their color charge property, causing them to be permanently bound inside hadrons (a composite subatomic particle made of two or more quarks \cite{CERN_hadron_tag}). As a result, jets are used as physical proxies for partons and are reconstructed using various algorithms to different types of objects.

In Chapter \ref{fifthchapter}, a selection of b-tagged jets is applied for the AGC. Jets that are "b-tagged" are matched to bottom and anti-bottom quarks using a flavour tagging algorithm in the pre-analysis of the data. The output of the algorithm are the $p_b$, $p_c$, and $p_u$ variables that are combined in Equation \ref{eq: btag}, where $f_c$ is a constant equal to $f_c = 0.018$. The final b-tagging discriminate is defined as $D_{DL1}$. A jet is considered as b-tagged if $D_{DL1}$ is above the threshold value of 2.456, corresponding to an efficiency of 77\% \cite{ATLAS_FTAG_r22prelim}. 
\begin{equation}
D_{DL1} = \log{(\frac{p_b}{f_c \times p_c + (1-f_c)\times p_u})}
\label{eq: btag}
\end{equation}
%%% Local Variables: 
%%% TeX-master: "mythesis"
%%% End: 


