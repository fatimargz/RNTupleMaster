\chapter{The ATLAS experiment}		% chapter 1
\label{secondchap}		% for reference (\ref{introchap})

% General introduction of ATLAS
In October 1992, the ATLAS collaboration, then composed of about 800 members, submited a letter of intent to the LHC Experiment Committee highlighting the design of what came to be today's ATLAS Experiment \cite{CERN:1992jvx}. From the start, ATLAS was designed to be a general-purpose experiment, optimized to search for the Higgs boson, top quark decays, and supersymmetry. In July 1997, the ATLAS Experiment was approved and by November 2008, ATLAS was the largest detector ever constructed at 44 meters long and 25 meters in diameter. By November 2009, ATLAS recorded its first proton-proton collision and by December 2010, ATLAS was first to observed the production of top quark pairs, which are the heaviest known elementary particle with a strong coupling to the Higgs boson. By July 2012, both ATLAS and the CMS Experiment successfully observed the infamous Higgs boson. ATLAS is projected to continue operation until 2035 to continue searching for standing questions from the SM. This chapter will serve as an introduction to the LHC and the ATLAS experiment.

\section{The Large Hadron Collider}


\section{The ATLAS Apparatus}
To do this, ATLAS has six different detecting subsystems wrapped concentrically in layers around the collision point to record the trajectory, momentum, and energy of particles. Apart, a hurge magnet system bends the paths of the charged particles so that their momenta can be measured as precisely as possible. Overall, the detector tracks and identifies particles to investigate a wide range of physics. 



\subsection{Inner Detector}
what is its main functions
\subsubsection{Pixel Detector}
\subsubsection{Semiconductor Tracker}
\subsubsection{Transition Radiation Tracker}

\subsection{Calorimeter}
\subsubsection{Liquid Argon Calorimeter}
\subsection{Tile Hadronic Calorimeter}

\subsection{Muon Spectrometer}
\subsubsection{Thin Gap Chambers}
\subsubsection{Resistive Plate Chambers}
\subsubsection{Monitored Drift Tubes}
\subsubsection{Small-Strip Thin-Gap}
\subsubsection{Micromegas}

\subsection{Magnet System}
\subsubsection{Central Solenoid Magnet}
\subsubsection{Barrel Toroid}
\subsubsection{End-cap Toroids} 

\section{ATLAS Trigger System}
\subsection{First-Level Hardware Trigger}
\subsection{Second-Level Hardware Trigger}


%%%%%%%%%%%%%%%%%%%%%%%%%%%%%%%%%%%%%%%%%%%%%%%%%%%%%%%%%%%%%%%%%%

%%% Local Variables: 
%%% TeX-master: "mythesis"
%%% End: 