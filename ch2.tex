\chapter{The ATLAS Experiment}		% chapter 1
\label{firstchap}		% for reference (\ref{introchap})

A Toroidal LHC Apparatus, mainly known as ATLAS, is a particle physics experiment located at the Large Hadron Collider near Geneva, Switzerland. ATLAS is 44 meters long and 25 meters in diameter, making it the largest detector ever constructed for a particle collider. It is designed as a general-purpose particle physics experiment, focused on precision measurements of the Standard Model. Research from ATLAS physicists has led to ground-breaking discoveries, such that of the Higgs boson. 

\section{Detector Technology}
To do this, ATLAS has six different detecting subsystems wrapped concentrically in layers around the collision point to record the trajectory, momentum, and energy of particles. Apart, a hurge magnet system bends the paths of the charged particles so that their momenta can be measured as precisely as possible. Overall, the detector tracks and identifies particles to investigate a wide range of physics. 



\subsection{Inner Detector}
what is its main functions
\subsubsection{Pixel Detector}
\subsubsection{Semiconductor Tracker}
\subsubsection{Transition Radiation Tracker}

\subsection{Calorimeter}
\subsubsection{Liquid Argon Calorimeter}
\subsection{Tile Hadronic Calorimeter}

\subsection{Muon Spectrometer}
\subsubsection{Thin Gap Chambers}
\subsubsection{Resistive Plate Chambers}
\subsubsection{Monitored Drift Tubes}
\subsubsection{Small-Strip Thin-Gap}
\subsubsection{Micromegas}

\subsection{Magnet System}
\subsubsection{Central Solenoid Magnet}
\subsubsection{Barrel Toroid}
\subsubsection{End-cap Toroids} 

\section{ATLAS Trigger System}
\subsection{First-Level Hardware Trigger}
\subsection{Second-Level Hardware Trigger}


%%%%%%%%%%%%%%%%%%%%%%%%%%%%%%%%%%%%%%%%%%%%%%%%%%%%%%%%%%%%%%%%%%

%%% Local Variables: 
%%% TeX-master: "mythesis"
%%% End: 