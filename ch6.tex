RNTuple performance meets expectations, demonstrating improvements in reading and writing speed and disk space usage, with little variance on memory consumption. An average size reduction of 47\% was observed when converting the TTree inputs to RNTuples using ZSTD compression algorithm. Using RDataFrame in C++, the average time execution for loading one column of data was found to be 2.4 times faster using RNTuple inputs versus TTree. The average time execution for writing a set with two columns was about 1.5 times faster for RNTuple than TTree. Through writing these outputs, it was revealed that RNTuple compresses repeated bits more efficiently than TTree. This opens up a potential new analysis approach that allows padding outputs with empty events. The peak memory usage measured while writing the two column output also improved by an average of 4,767 bytes. These improvements were achieved while preserving the RDataFrame workflow in C++ and requiring only minimal code changes, demonstrating a seamless transition from TTree to RNTuple.

An initial evaluation of RNTuple behavior using the \texttt{ZSTD} and \texttt{LZ4} compression algorithms was conducted. Performance tests repeated with RNTuple inputs produced using the \texttt{LZ4} compression algorithm show a 14\% file size increase with no significant improvements in reading or writing speeds. Inputs produced with \texttt{LZ4} exhibited only millisecond-level improvements when loading a field compared to RNTuple inputs produced with \texttt{ZSTD}. Writing speed also improved slightly, by about two seconds when using \texttt{LZ4}-produced RNTuple inputs versus those produced with \texttt{ZSTD} inputs. These minimal improvements in reading and writing speeds occur at the cost of larger file sizes. Given these result, the \texttt{ZSTD} compression algorithm is recommended when producing RNTuples. 

The implementations of the AGC using ATLAS Open Data and RDataFrame were completed for both TTree and RNTuple inputs. The RNTuple version represents the first full end-to-end implementation of a physics analysis using RNTuple and demonstrates the feasibility and improvements of the format. The analysis script for the RNTuple AGC version remained largely unchanged from the TTree version, indicating a smooth analysis workflow transition within RDataFrame. For completeness, performance studies were repeated using the AGC and demonstrate consistent RNTuple performance. The total execution times for reading the 92 ATLAS Open Data inputs and writing the top mass and $H_T$ histograms were improved by an average of about 24.17 seconds using RNTuple inputs; hence, the RNTuple version was 1.5 times faster than the TTree version. No significant improvements in speed were observed when executing the AGC with RNTuple inputs produced with \texttt{LZ4}. Overall, these consistent improvements in RNTuple performance reaffirms it as a promising data storage format for ATLAS analysis using RDataFrame workflows. Its minimal code modifications also promise a smooth transition toward the HL-LHC.
