RNTuple performance meets expectations, demonstrating improvements in reading and writing speed and disk space usage, with little variance on memory consumption. Using RDataFrame in C++, it was shown that the average time execution for loading one column of data is about 2.4 times faster for RNTuple than TTree. The average time execution for writing a set with two column was about 1.5 times faster for RNTuple than TTree. An average size reduction of 47\% was also observed when converting the TTree inputs to RNTuples using ZSTD compression algorithm. The peak memory usage when measured while writing the two column output also improved by an average of 4,767 MaxRSS. 

Performance tests repeated for RNTuple inputs produced with the LZ4 compression algorthims demonstrate an increase of 14\% file size with no significant improvements in reading and writing speeds. The inputs produced with LZ4 showed an improvement in the order of milliseconds when loading a field in comparison to RNTuple inputs produced with ZSTD. Writing speed also improved, but by about two seconds with LZ4 RNTuple inputs versus ZSTD inputs.  

The RDataFrame AGC for ATLAS Open Data is complete for both TTree and RNTuple versions, signaling a strong start for RNTuple implementation for the HL-LHC. The analysis script for the RNTuple AGC version remained mostly unmodified to the TTree version, indicating a smooth analysis workflow transition for RDataFrame. Performance studies were repeated using the AGC version for completion and demonstrate consistency in RNTuple performance. The total execution times for reading the 91 ATLAS Open Data inputs and writing the top mass and Ht histograms were improved by an average of about 24.17 seconds using RNTuple inputs; hence, the RNTuple version was 1.5 times faster than the TTree version. No significant improvements in speed were observed when executing the AGC with RNTuple inputs produced with LZ4.

