\chapter{ATLAS Software and Computing}		
\label{thirdchapter}

% introduction of the different data we collect 
The data collected from the ATLAS data acquisition system must be compared to a set of simulated data. This dataset aims to mimic the different physics processes: it's production by the colliding beams, the evolution of the collision products within the detector and materials, and the detector's response to ultimately interpret efficiencies and background processes. Except for collision data, the output of all these data processing steps are stored in ROOT files. It starts off with Monte Carlo (MC) simulations, which is a computational technique that uses random sampling to generate events. Given these events, the interactions within the detector and the detector's response is simulated. This reconstructed product is called an Analysis Object Data (AOD), which are then cleaned by compressing the data and cutting any unnecessary events or columns into a finalized product called Derived AOD (DAOD). At each step, each produced ROOT file is validated by dedicated tools. These tools collectively encompass the software framework call Athena \cite{ATHENA}. The flow of this process is display in Figure \ref{fig:DataChain}. This chapter will provide an introduction to ROOT and its data storage format called TTree.
\begin{figure}
\centerline{\includegraphics[height=95mm]{ch3_images/datachain.png}}
\caption[DataChain]{ATLAS data chain-processing for data and Monte Carlo simulation \cite{JCatmore}.}
\label{fig:DataChain}
\end{figure}

\section{ROOT Introduction}
% history of ROOT
ROOT is a unified software package developed for processing, analyzing, visualizing and ultimately storing the massive high-energy physics datasets into a compressed binary file, called a root file. Previously, high-energy experiments used FORTRAN-based libraries; however, an upgrade was needed to handle the scales and complexities of the data from the LHC. ROOT maintains an object-oriented structure, meaning it is organized around the data rather than the functions and logic. It's features include visualization tools such as histogramming, and statistical tools. ROOT can be used in C++ and python languages. Several subpackages exists for analysis such as RDataFrame and uproot. 

\subsection{TTree Introduction}
ROOT provides a data structure called the TTree to store large amounts of columnar data efficiently. Usually scientific data is stored in what we call row-oriented formats such as a spreadsheet or CSV table. This format is well organized if one wants to access a single event, but viewing a single column then becomes inefficient, especially with large datasets. A TTree is columnar based, meaning it consists of a list of independent columns, called branches. Examples of branches can be event IDs or particle kinematics such as momentum in the x,y,z coordinates. Branches can hold integers, strings and std::vector data types. Buffers are automatically allocated behind each branch. Buffers are temporary storage areas for the independent binary version of the object. This is done to efficiently handle the writing and reading of the data to and from disk. Also, each branch has one or more baskets, which manages the in-memory buffer. In other words, a basket holds the values of a branch for a number of consecutive events. When a buffer is full, it is optionally compressed and then the corresponding basket is written to disk, leading to the creation of a new basket to hold the next entries. ROOT allows users to change buffersize parameters of the branch for personalized optimization. Figure \ref{fig:TTreeDataStructure} shows a more detailed flowchart of the TTree data structure.
\begin{figure}
\centerline{\includegraphics[height=130mm, width=.65\textwidth]{ch3_images/TTreeDataStructure.png}}
\caption[TTree Data Structure]{Example of the TTree Data Structure \cite{ROOTTTree}.}
\label{fig:TTreeDataStructure}
\end{figure}
% history of TTree 
