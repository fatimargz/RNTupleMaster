The data collected from the detector must be compared to a set of simulated data in order to interpret efficiencies and background processes. These data sets aim to mimic different physics processes such as the events produced by the collider beams, the evolution of the collision products within the detector and materials, and the detector's response. The preparation of simulation starts off with Monte Carlo (MC) simulation, which is a computational technique that uses random sampling to generate events. Given these events, the interactions within the detector and the detector's response are simulated. This reconstructed product is called an Analysis Object Data (AOD), which are then cleaned by compressing the data and cutting any unnecessary events or columns into a finalized product called Derived AOD (DAOD). The products produced at each step are then stored into a compressed binary file, called a ROOT file, and are validated using different software tools. The software framework that encompasses the tools needed to produce, validate, and analyze all of these types of samples is called Athena \cite{ATLAS_Athena_Zenodo_3932810}. The flow of this process, compared to the similar process followed by collision data, is shown in Figure \ref{fig:DataChain}. 

\begin{figure}[ht]
\centerline{\includegraphics[height=95mm]{ch3_images/datachain.png}}
\caption[Data Chain]{ATLAS data chain-processing for data and Monte Carlo simulation \cite{ATLAS_Catmore_2020_DataProcessingChain}.}
\label{fig:DataChain}
\end{figure}

The studies for this thesis use data at the analysis level. This chapter will introduce ATLAS Open Data \cite{ATLAS_OpenData_DAODPHYSLITE_2015_2016}, ROOT, and the data structures of TTree versus RNTuple. It will also provide a comparison between the application programming interface (API) for the TTree and RNTuple formats.

\section{ATLAS Open Data}\label{sec:opendata}
ATLAS Open Data is a publicly available dataset produced by the ATLAS collaboration. It's composed of MC simulations of particle collisions within the ATLAS detector and detector data measurements. The data used as inputs for the remainder of this study are MC simulations of top samples from Run 2 \cite{ATLAS_OpenData_DAODPHYSLITE_MC_2024}. They are simulated samples of single top quarks, matter-antimatter $t\overline{t}$ pairs, and W boson production in association with jets. Representative diagrams for these processes are shown in Figure \ref{fig: samples}.

\begin{figure}
\centerline{\includegraphics[width=.8\textwidth]{ch3_images/samples.png}}
\caption[Data Chain]{The representative diagrams of the processes used in this thesis.}
\label{fig: samples}
\end{figure}

The inputs are all provided in DAOD\_PHYSLITE format, which contains already-calibrated objects directly from an AOD or PHYS product \cite{ATLAS_Collaboration_2023_PHYSLITE}. Those objects include jets, electrons, muons, photons, taus and their properties, such as momentum, mass, charge, eta, and phi. Each event contains a number of physical objects that depends on the underlying process, resulting in a multidimensional dataset. A full description of the variables can be found in \cite{ATLAS_OpenData_PHYSLITE_Docs}.

\section{ROOT}
% history of ROOT
ROOT is a unified software package developed for processing, analyzing, visualizing and ultimately storing the massive high-energy physics datasets. Previously, high-energy experiments used FORTRAN-based libraries; however, an upgrade was needed to handle the scales and complexities of the data from the LHC \cite{Brun:2296392}. ROOT maintains an object-oriented structure, meaning it is organized around the data rather than the functions and logic. Its features include visualization tools such as histogramming, and statistical tools. ROOT can be used in C++ and Python languages. ROOT’s declarative analysis interface (RDataFrame \cite{ROOT_RDataFrame_class}) is used extensively in this thesis, both in Python and in C++.

\subsection{ROOT Compression Algorithms}
ROOT offers four different compression algorithms: \texttt{ZLIB}, \texttt{LZMA}, \texttt{LZ4}, and \texttt{ZSTD} \cite{Marcon:2024zsm}. Data compression allows users to store large files at reduced sizes without losing information from the original file. It can also increase data reading and writing speeds. There are generally two types of compression algorithms: lossless and lossy. Lossy algorithms can reduce file sizes with larger compression factors, but are irreversible processes, meaning that information is lost during compression. The four compression algorithms from ROOT are lossless algorithms, meaning they are reversible processes that reduce bits by eliminating statistical redundancy.  

There are advantages and disadvantages in each of the four algorithms. \texttt{LZ4} focuses on compression and decompression speed, yet offers smaller compression factors and thus larger file sizes. \texttt{LZMA} provides higher compression at the cost of significantly slower reading speeds. \texttt{ZLIB} is an older version of \texttt{ZSTD}. Both provide a balance between compression and reading speeds; however, \texttt{ZSTD} has been shown to perform better in all metrics in comparison to ZLIB \cite{BockelmanShadura_2021_ZstdLZ4}. The default compression algorithm used to generation RNTuple samples is \texttt{ZSTD}. In Chapter \ref{fourthchapter}, a performance study using RNTuples produced with \texttt{LZ4} is shown for comparison.

\subsection{TTree Data Structure}
ROOT provides a columnar-based data structure called TTree to efficiently store data. The columnar-based format allows users to access independent columns of data, such as event IDs or particle kinematics, versus accessing information per event. Internally, the physical data of each column are stored as Binary Large Objects (BLOBs), which are compressed blocks of serialized values. These BLOBs are stored in containers called TBaskets, each holding the data for a consecutive range of entries. Each column of data has a corresponding TBranch object that stores metadata. The metadata describes the column, such as the column's data type and serialization rules, and if the columns has subcomponents, which are represented by associated TLeaf objects. For I/O (Input/Output) efficiency, TTrees group consecutive entries into units called clusters. 

The TBranch also stores metadata containing the list of file offsets and sizes of all of its baskets. ROOT uses these offsets and sizes to implicitly address the TBaskets and ultimately, decompress it and return the corresponding column data to the user. Figure \ref{fig:TTreeDataStructure} shows a more detailed flowchart of the TTree data structure.
\begin{figure}[ht]
\centerline{\includegraphics[height=130mm, width=.65\textwidth]{ch3_images/TTreeDataStructure.png}}
\caption[TTree Data Structure]{Representation of the TTree Data Structure \cite{ROOT_TTree_v6-30}.}
\label{fig:TTreeDataStructure}
\end{figure}

\subsection{RNTuple Data Structure}
RNTuple is the new columnar data format that will be implemented at the start of the HL-LHC. Its design continues to be columnar based, as its predecessor TTree, but it now uses modern storage technologies for better performance characteristics in data compactness, scalability, and read and write speed. For this reason, RNTuple classes are backwards-incompatible to TTree both on the file format level and API level \cite{ROOT_RNTuple_BinaryFormatSpecification}. Its binary format version follows an \emph{epoch.major.minor.path} scheme, where \emph{epoch} indicates backward-incompatible changes, \emph{major} indicates forward-incompatible changes, \emph{minor} indicates new optional format features, and \emph{patch} indicates backported features from newer format versions. The remainder of this study uses the first public release of RNTuple 1.0.0.0.

RNTuple separates logical schema from physical layout explicitly. Unlike TTree, RNTuple has an \emph{anchor}, which is a top-level metadata object that describes the schema, columns, and how the physical data is stored. Each column has a corresponding \emph{field}, which is an RNTuple object that maps the column to its physical storage location. Each column's data is stored in fixed-size units called \emph{pages}. Pages are the containers that hold the actual serialized BLOBs. Pages are grouped into larger units called \emph{clusters}, which define the granularity at which metadata and I/O decisions are made. Each cluster contains an \emph{envelope}, which are metadata blocks describing the physical locations of all the pages in that cluster. Inside envelopes are \emph{RBLOB} keys, which are the lowest-level object that hold the physical offset and size of a single page or multiple pages of the same cluster. Using RBLOB keys, RNTuple can explicitly locate, read, and decompress pages back to the user. A flow chart of this process in comparison to TTree is shown in Figure \ref{fig:RNTupleFlow}

Overall, this structure allows for more efficient random-access of individual events in comparison to TTree. For TTree, metadata is spread-out and stored in each individual TBranch; therefore,  its access pattern is more tightly coupled to sequential iteration. For RNTuple, the separation of the logical schema and physical layout allows for explicit addressing of the data and improves data compactness. 

\begin{figure}[ht]
\centerline{\includegraphics[width=\textwidth,height=100mm]{ch3_images/TTreevsRNTupleDataStructures.png}}
\caption[TTree vs. RNTuple Mapping]{TTree versus RNTuple mapping systems.}
\label{fig:RNTupleFlow}
\end{figure}

\section{TTree vs. RNTuple API}
The first portion of this thesis aimed to test RNTuple API against TTree using ATLAS analysis workflows. Efforts were made to document the capabilities, usage and best practices of RNTuple during its development. The corresponding code repository, including the aforementioned documentation can be found in \cite{Rodriguez_2025_RNTupleWorkflows}. The sections below will provide examples of RNTuple's API in comparison to TTree's, using native C++ event loops, RDataFrame, and Uproot \cite{uproot}. 

\subsection{Native C++ Event Loops}
Due to the multidimensional nature of particle physics data, event loops are common algorithms used in data analysis workflows. It is a process that continuously iterates through the large datasets to apply specific analysis steps to each event. As seen in Figure \ref{fig:eventloop-ttree}, users must iterate through TTree in order to load branches and define an empty pointer object to store their entries.

The RNTuple interface uses smart pointers, which simulate a pointer while providing automatic memory management \cite{Smart_Pointer_Wikipedia}. This feature shortens the amount of code necessary to read and load data by a couple of lines. For example \texttt{RNTupleReader::Open} simultaneously loads the ROOT file and the RNTuple, as seen in Figure \ref{fig:eventloop-rnt}. The function \texttt{GetView} also simultaneously loads and stores a field.

\begin{figure}[ht]
\begin{subfigure}{\linewidth}
\centerline{\includegraphics[width=.8\textwidth,height=100mm]{ch3_images/ttree.png}}
\caption{Native C++ event loop using TTree.}
\label{fig:eventloop-ttree}
\end{subfigure}
\begin{subfigure}{\linewidth}
\centerline{\includegraphics[width=\textwidth,height=90mm]{ch3_images/rntuple.png}}
\caption{Native C++ event loop using RNTuple.}
\label{fig:eventloop-rnt}
\end{subfigure}
\caption{These scripts load a DAOD\_PHYSLITE file containing a TTree in (a) and an RNTuple in (b) to plot the distribution of electron transverse momenta. Transverse momentum is the momentum perpendicular of the colliding beams.}
\end{figure}

\subsection{RDataFrame in C++ and Python}
RDataFrame is a high-level interface for analysis of data stored in TTree, RNTuple, CSV, and other data formats. RDataFrame can be used via C++ or Python languages. Analysis done with RDataFrame will mostly remain unmodified with RNTuple, as shown in Figure \ref{fig:rdfpython}, with the exception of filtering. Due to RNTuple's internal data structure, sub fields such as "\texttt{AnalysisElectronsAuxDyn:pt}" are separated by their field, "\texttt{AnalysisElectronsAuxDyn}" by a colon, instead of a period. This slight change confuses the filtering function in RDataFrame, but can be bypassed by assigning an alias name. Figure \ref{fig:rdfc} provides an example of how to read multiple inputs and apply a filter using RDataFrame in C++.

\begin{figure}[ht]
\begin{subfigure}{\linewidth}
\centerline{\includegraphics[width = 0.8\linewidth]{ch3_images/ttree-rdf-python.png}}
\caption{Reading multiple TTree inputs.}
\end{subfigure}
\begin{subfigure}{\linewidth}
\centerline{\includegraphics[width=0.8\linewidth]{ch3_images/rnt-rdf-python.png}}
\caption{Reading multiple RNTuple inputs.}
\end{subfigure}
\caption[Reading Multiple Inputs in RDataFrame Python: TTree vs. RNTuple]{Examples of how to load multiple inputs into an RDataFrame in Python.}
\label{fig:rdfpython}
\end{figure}

\begin{figure}[ht]
\begin{subfigure}{\linewidth}
\centerline{\includegraphics[width = 0.8\linewidth]{ch3_images/ttree-rdf.png}}
\caption{Reading multiple TTree inputs.}
\end{subfigure}
\begin{subfigure}{\linewidth}
\centerline{\includegraphics[width=0.8\linewidth]{ch3_images/rnt-rdf.png}}
\caption{Reading multiple RNTuple inputs.}
\end{subfigure}
\caption[Applying an RDataFrame Filter in C++ Using Multiple Inputs]{Examples of how to load multiple inputs into an RDataFrame and create a new filtered dataframe in C++.}
\label{fig:rdfc}
\end{figure}

\subsection{Uproot}
Uproot is a Python library designed to extract and manage data from ROOT files. Analysis done with Uproot will also remain unmodified with RNTuple. Users can load RNTuples and convert them into dictionaries of arrays just as with TTrees. Also, Uproot's added function, \texttt{uproot.mkrntuple}, allows users to write RNTuple outputs. Loading multiple RNTuple inputs using uproot's functions \texttt{uproot.concatenate} and \texttt{uproot.iterate} also work with RNTuple. 


